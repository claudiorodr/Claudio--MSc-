\chapter*{Abstract}
\justify
With the surge of inexpensive, widely accessible, and precise microelectromechanical systems (MEMS) in recent years, inertial systems tracking movement have become ubiquitous nowadays. Contrary to GPS-based positioning, Inertial Navigation Systems (INS) are intrinsically unaffected by signal jamming, blockage susceptibilities, and spoofing. Measurements from inertial sensors are also acquired at elevated sampling rates and may be numerically integrated to estimate position and orientation knowledge. These measurements are precise on a small-time scale but gradually accumulate errors over extended periods. Combining multiple inertial sensors in a method known as sensor fusion makes it possible to produce a more consistent and dependable understanding of the system, decreasing accumulative errors. Several sensor fusion algorithms occur in literature aimed at estimating the Attitude and Heading Reference System (AHRS) of a rigid body with respect to a reference frame. This work describes the development and implementation of a low-cost, multipurpose INS for position and orientation estimation. It presents an experimental comparison of the most popular sensor fusion solutions. Additionally, the study explores the prospect of integrating transmission Internet of Things (IoT) devices (based on an open radio frequency at 868 MHz) to broadcast in real-time and at a long distance the navigational data of a moving object and how this approach may be employed in a series of real-world scenarios.

\keywords{Sensor Fusion  \and Inertial Navigation System (INS) \and Microelectromechanical system (MEMS) \and Internet of Things (IoT) \and Inertial Measurement Unit (IMU) \and Aerial Assessments }