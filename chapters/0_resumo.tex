\chapter*{Resumo}
\justify
Com o aumento de microelécânicos de sistemas baratos, amplamente acessíveis e precisos nos últimos anos, o movimento de rastreio de sistemas de inércia tornou-se omnipresente nos dias de hoje. Contrariamente à localização baseada no Sistema de Posicionamento Global (GPS), os Sistemas de Navegação Inercial (SNI) não são afetados intrinsecamente pelo bloqueio de sinais, pelo bloqueio de suscetibilidades e pela falsificação. As medições dos sensores de inércia também são adquiridas a taxas elevadas de amostragem e podem ser integradas numericamente para estimar os conhecimentos de posição e orientação. Estas medições são precisas numa escala de pequena dimensão, mas acumulam gradualmente erros durante longos períodos. Combinar múltiplos sensores de inércia num método conhecido como fusão de sensores permite produzir uma compreensão mais consistente e fiável do sistema, diminuindo os erros acumulados. Vários algoritmos de fusão de sensores ocorrem na literatura com o objetivo de estimar os Sistemas de Referência de Atitude e Rumo de um corpo rígido no que diz respeito a um quadro de referência. Este trabalho descreve o desenvolvimento e implementação de um Sistemas multiusos de baixo custo para estimativa de posição e orientação. Além disso, apresenta uma comparação experimental de uma série de soluções de fusão de sensores e compara o seu desempenho na estimativa da posição de um objeto em movimento.

\keywords{ Fusão de Sensores  \and Sistemas de navegação Inercial \and Sistemas Microeletromecânicos \and Unidade de Medição Inercial \and Sistemas de Referência de Atutude e Rumo }
