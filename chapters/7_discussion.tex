\section{Discussion}
Facing the result from the previous chapter, a few conclusions can be drawn. Generally, the results showed a good estimation accuracy by most sensor fusion algorithms, with small difference between the best performing algorithms. The results showed that the algorithms had generally smaller displacement error than turn error which indicates that algorithms were generally better at estimating position than necessarily the shape of the experiment. Another conclusion was that absolute error increases with distance, since longer experiments had bigger absolute error rates than shorter experiments. Relative error measurements, on the other side, decreased in longer tests. Some unexpected surprises from less known algorithms such as OLEQ, ROLEQ, SAAM, Tilt, FLAE and Davenport outperforming more well-established algorithms like the EKF and Madgwick. From every test made, ROLEQ ranks best at both displacement and turn estimation with an average of  meters of displacement error and  meters of turn error. Closely followed by OLEQ and Mahony with 0.79 meters of displacement error, and 1.02 and 0.97 meters of turn error respectively. On the other hand, Fourati had the worst general performance among all algorithms, with a 7.04 meter error average displacement and 8.32 meter turn error average. Angular Rate, FAMC and AQUA also had poor overall results generally performing under average of the other fusion algorithms.
\vspace{-0.5cm}
\begin{table}[H]
    \begin{center}
        \resizebox{1\linewidth}{!}{
            \begin{tabular}[t]{lcccc}
                \hline
                Algorithm     & Displacement Error[$m$] & Displacement Error[\%] & Turn Error[$m$] & Turn Error[\%] \\
                \hline
                AngularRate   & 4.50                    & 37.47                  & 6.51            & 54.28          \\
                AQUA          & 2.91                    & 24.24                  & 5.66            & 47.18          \\
                Complementary & 0.85                    & 7.08                   & 1.46            & 12.14          \\
                Davenport     & 0.93                    & 7.75                   & 1.36            & 11.35          \\
                EKF           & 1.02                    & 8.50                   & 0.92            & 7.66           \\
                FAMC          & 4.23                    & 35.23                  & 6.19            & 51.56          \\
                FLAE          & 0.91                    & 7.54                   & 1.30            & 10.82          \\
                Fourati       & 7.04                    & 58.67                  & 8.32            & 69.35          \\
                Madgwick      & 1.68                    & 14.03                  & 2.00            & 16.63          \\
                Mahony        & 0.79                    & 6.60                   & 0.97            & 8.11           \\
                OLEQ          & 0.79                    & 6.57                   & 1.02            & 8.53           \\
                QUEST         & 2.46                    & 20.54                  & 3.30            & 27.49          \\
                ROLEQ         & 0.78                    & 6.54                   & 0.84            & 6.98           \\
                SAAM          & 0.87                    & 7.24                   & 1.10            & 9.16           \\
                Tilt          & 0.87                    & 7.24                   & 1.10            & 9.16           \\
                \hline
                Average       & 2.04                    & 17.02                  & 2.80            & 23.36
            \end{tabular}
        }
        \caption{ Overall ranking of how every algorithm performed in the given test. }
        \label{tab:overall}
    \end{center}
\end{table}
\vspace{-1.5cm}
To develop and determine an accurate positioning system is a challenge on many levels as it requires computationally advanced electronic components that can perform highly accurate calculations. The difficulties faced when solving this problem arise when the determining the orientation of an object as it produces many sources of error. The downside of having a well oriented system is at the expense of detection of low velocities and vice versa. The positioning system performs at high accuracy (cm precision) during non-rotational movements but has trouble with drift when the object is rotated. This is due to the low convergence rate of the orientation filter which is clearly a big system limitation. A suggestion on solving this problem could be making the tuning parameters of the filter more dynamic which would make the system more adaptable to extreme rotations and low velocities.

Common problems faced when trying to implement an accurate position lies in the gyroscope because it tends to drift significantly over time. The observations made during the studying of other positioning systems have been that they all suffer in accuracy due the gyroscope drift. These findings lead to the consideration of reducing the gyroscope drift significantly. Methods and algorithms have been presented in this thesis which evidently reduced the drift components and these methods have helped increase the accuracy of the positioning system. Similarly, to other systems, this system suffers from inaccuracies. Due to the gyroscope drift reduction algorithms developed, this system stands out from other previously developed systems in that sense. To develop a very accurate positioning system using low-cost MEMS sensors is a challenging subject that comes with many complications. The system requirements are very strict due to the environmental, economical and accuracy specifications. Therefore, implementing a system of very high accuracy, low cost and high mobility is very difficult. To achieve a higher accuracy, the electronic components need to have a higher performance, higher resolution, and a higher sampling frequency. The micro controller unit which is the heart of the whole system needs to have a faster processor but, in the end, it is the sensors that limit the system's performance.
