\section{Discussion}
Facing the result from the previous chapter, a few conclusions can be drawn. Generally, the results showed a good estimation accuracy by most sensor fusion algorithms, with small difference between the best performing algorithms (table \ref{tab:overall}).

% \vspace{-0.5cm}
\begin{table}[H]
    \begin{center}
        \resizebox{1\linewidth}{!}{
            \begin{tabular}[t]{lcccc}
                \hline
                Algorithm     & Displacement Error[$m$] & Displacement Error[\%] & Turn Error[$m$] & Turn Error[\%] \\
                \hline
                AngularRate   & 4.50                    & 37.47                  & 6.51            & 54.28          \\
                AQUA          & 2.91                    & 24.24                  & 5.66            & 47.18          \\
                Complementary & 0.85                    & 7.08                   & 1.46            & 12.14          \\
                Davenport     & 0.93                    & 7.75                   & 1.36            & 11.35          \\
                EKF           & 1.02                    & 8.50                   & 0.92            & 7.66           \\
                FAMC          & 4.23                    & 35.23                  & 6.19            & 51.56          \\
                FLAE          & 0.91                    & 7.54                   & 1.30            & 10.82          \\
                Fourati       & 7.04                    & 58.67                  & 8.32            & 69.35          \\
                Madgwick      & 1.68                    & 14.03                  & 2.00            & 16.63          \\
                Mahony        & 0.79                    & 6.60                   & 0.97            & 8.11           \\
                OLEQ          & 0.79                    & 6.57                   & 1.02            & 8.53           \\
                QUEST         & 2.46                    & 20.54                  & 3.30            & 27.49          \\
                ROLEQ         & 0.78                    & 6.54                   & 0.84            & 6.98           \\
                SAAM          & 0.87                    & 7.24                   & 1.10            & 9.16           \\
                Tilt          & 0.87                    & 7.24                   & 1.10            & 9.16           \\
                \hline
                Average       & 2.04                    & 17.02                  & 2.80            & 23.36
            \end{tabular}
        }
        \caption{ Overall ranking of how every algorithm performed in the given test. }
        \label{tab:overall}
    \end{center}
\end{table}
% \vspace{-1.5cm}

The results showed that the algorithms had smaller displacement error than turn error which indicates that algorithms were generally better at estimating position than necessarily the shape of the experiment. Another conclusion was that absolute error increases with distance, since longer experiments had bigger absolute error rates than shorter experiments. Relative error measurements, on the other side, decreased in longer tests. Some unexpected surprises from less known algorithms such as \acrshort{oleq}, \acrshort{roleq}, \acrshort{saam}, Tilt, \acrshort{flae} and Davenport outperforming more well-established algorithms like the Complementary, \acrshort{ekf} and Madgwick.

From every test made, Mahony ranks best at displacement and with an average of 1.94 meters of absolute displacement error and 3.72\% relative error. Closely followed by \acrshort{saam} and Tilt with 2 meters of displacement error, and 4.12\% of relative displacement error (table \ref{tab:ranked_displacement}).

% \vspace{-0.5cm}
\begin{table}[H]
    \begin{center}
        % \resizebox{1\linewidth}{!}{
        \begin{tabular}[t]{lcccc}
            \hline
            Algorithm        & Displacement Error[$m$] & Displacement Error[\%] \\
            \hline
            Mahony           & 1.94                    & 3.72                   \\
            \acrshort{saam}  & 2                       & 4.12                   \\
            Tilt             & 2                       & 4.12                   \\
            Davenport        & 2.27                    & 4.96                   \\
            \acrshort{flae}  & 2.42                    & 5.02                   \\
            \acrshort{ekf}   & 2.59                    & 6.71                   \\
            \acrshort{oleq}  & 2.65                    & 4.65                   \\
            \acrshort{roleq} & 2.71                    & 4.92                   \\
            Madgwick         & 3.86                    & 8.35                   \\
            Complementary    & 5.49                    & 10.28                  \\
            \acrshort{aqua}  & 6.45                    & 18.51                  \\
            \acrshort{quest} & 10.69                   & 22.55                  \\
            \acrshort{famc}  & 15.43                   & 22.11                  \\
            AngularRate      & 16.35                   & 23.53                  \\
            Fourati          & 24.19                   & 36.14                  \\
        \end{tabular}
        % }
        \caption{ Algorithms ranked according to of how well performed at displacement error. }
        \label{tab:ranked_displacement}
    \end{center}
\end{table}
% \vspace{-1.5cm}

\acrshort{saam} and Tilt ranked best at turn error both with an average of 2.79 meters of absolute turn error and 6.93\% of relative turn error. Closely followed by Mahony with 6.97\% meters of relative turn error (table \ref{tab:ranked_turn}).

% \vspace{-0.5cm}
\begin{table}[H]
    \begin{center}
        % \resizebox{1\linewidth}{!}{
        \begin{tabular}[t]{lcccc}
            \hline
            Algorithm        & Turn Error[$m$] & Turn Error[\%] \\
            \hline
            \acrshort{saam}  & 2.79            & 6.93           \\
            Tilt             & 2.79            & 6.93           \\
            Mahony           & 2.93            & 6.97           \\
            Davenport        & 3.01            & 7.95           \\
            \acrshort{oleq}  & 3.17            & 7.99           \\
            \acrshort{flae}  & 3.19            & 7.96           \\
            \acrshort{ekf}   & 3.21            & 9.27           \\
            \acrshort{roleq} & 3.35            & 7.92           \\
            Madgwick         & 4.64            & 10.42          \\
            Complementary    & 6.72            & 13.44          \\
            \acrshort{aqua}  & 9.15            & 26.54          \\
            \acrshort{quest} & 16.34           & 30.34          \\
            \acrshort{famc}  & 22.57           & 35.22          \\
            AngularRate      & 25.96           & 39.64          \\
            Fourati          & 31.87           & 48.16          \\
        \end{tabular}
        % }
        \caption{ Algorithms ranked according to of how well performed at turn error. }
        \label{tab:ranked_turn}
    \end{center}
\end{table}
% \vspace{-1.5cm}

On the other hand, Fourati had the worst general performance among all algorithms, with a 24.19 (36.14\% relative displacement error) meter error average displacement and 31.87 (48.16\% relative turn error) meter turn error average. Angular Rate, \acrshort{famc}, \acrshort{quest} and \acrshort{aqua} also had poor overall results generally performing under average of the other fusion algorithms.

From table \ref{tab:ahrs_algorithms}, it is possible to draw a correlation between what sensor are used by each algorithm and how they performed at estimating position. Tables \ref{tab:ranked_displacement} and \ref{tab:ranked_turn} show that the worst performing algorithms are generally the ones who don't use every sensor available, for instance \acrshort{famc}, \acrshort{quest} and \acrshort{aqua} don't fuse the gyro's readings to estimate orientation, while Angular Rate uses only the gyro's measurements. Fourati's poor performance might be explained by the fact that it depicts rigid body's attitude in space with respect to the navigation frame (YN,YN,ZN), where the navigation frame follows the convention \acrfull{ned}, opposite of the convention used by the other algorithms, \acrfull{enu}.

To develop and determine an accurate positioning system is a challenge on many levels as it requires computationally advanced electronic components that can perform highly accurate calculations. The difficulties faced when solving this problem arise when the determining the orientation of an object as it produces many sources of error. The downside of having a well oriented system is at the expense of detection of low velocities and vice versa. The positioning system performs at high accuracy (cm precision) during non-rotational movements but has trouble with drift when the object is rotated. This is due to the low convergence rate of the orientation filter which is clearly a big system limitation. A suggestion on solving this problem could be making the tuning parameters of the filter more dynamic which would make the system more adaptable to extreme rotations and low velocities.

Common difficulties encountered when attempting to employ an accurate position estimation system lied significantly in the sensor’s output reliability and consistency since these types of electronic devices are highly susceptible to the existence of errors. Ever since the beginning of the development of this project we realized calibration would play an important role if this idea was to ever take off. The observations made throughout the examining of other positioning systems have been that calibration ensures accurate measurements, and accurate measurements are the foundation to the quality of data. These discoveries lead to the significance of calibration as a means of reducing inaccuracies and increase data quality significantly. As seen in chapter \ref{sub:calibration} \nameref{sub:calibration}, several methods and algorithms have been established which evidently decreased the drift and noise components and have helped increase the accuracy of the positioning system. To develop a very accurate positioning system using low-cost \acrshort{mems} sensors is a challenging subject that comes with many complications. The system requirements are very strict due to the environmental, economical and accuracy specifications. Therefore, implementing a system of very high accuracy, low cost and high mobility is very difficult. To achieve a higher accuracy, the electronic components need to have a higher performance, higher resolution, and a higher sampling frequency. The micro controller unit which is the heart of the whole system needs to have a faster processor but, in the end, it is the sensors that limit the system's performance.
